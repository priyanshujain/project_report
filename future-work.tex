\chapter{Future Work}

Conversational modeling is an important task in
natural language understanding and machine in-
telligence. Although previous approaches that is implemented
, that are often restricted to specific domains
and require hand-
crafted rules. Now I will move to a sim-
ple approach for this task which uses the recently
proposed sequence to sequence framework.\\[1.0cm]

\section{seq2seq framework}
This approach is based on recent work which pro-
posed to use neural networks to map sequences to se-
quences (Kalchbrenner \& Blunsom, 2013; Sutskever et al.,
2014; Bahdanau et al., 2014). This
model converses by predicting the next sentence
given the previous sentence or sentences in a
conversation. The strength of This model is that
it can be trained end-to-end and thus requires
much fewer hand-crafted rules. We find that this
straightforward model can generate simple con-
versations given a large conversational training
dataset. Our preliminary suggest that, despite op-
timizing the wrong objective function, the model
is able to extract knowledge from both a domain
specific dataset, and from a large, noisy, and gen-
eral domain dataset of movie subtitles. This is implemented by google brain team On a
domain-specific IT helpdesk dataset, the model
can find a solution to a technical problem via
conversations. On a noisy open-domain movie
transcript dataset, the model can perform simple
forms of common sense reasoning. As expected,
we also find that the lack of consistency is a com-
mon failure mode of our chatbot or conversational model.\\[1.0cm]

\begin{figure}[htb]
\centering
\includegraphics[scale=0.5]{neural} % e.g. insert ./image for image.png in the working directory, adjust scale as necessary
\caption{Neural modeling}
\label{1label} % insert suitable label, this is used to refer to a fig from within the text as shown above
\end{figure}
\begin{figure}[htb]
\centering
\includegraphics[scale=1.0]{neuralcm} % e.g. insert ./image for image.png in the working directory, adjust scale as necessary
\caption{Using the sequence to sequence framework for modeling conversations}
\label{1label} % insert suitable label, this is used to refer to a fig from within the text as shown above
\end{figure}
